\documentclass[prb,preprint]{revtex4-1} 
% The line above defines the type of LaTeX document.
% Note that AJP uses the same style as Phys. Rev. B (prb).

% The % character begins a comment, which continues to the end of the line.

\usepackage{amsmath}  % needed for \tfrac, \bmatrix, etc.
\usepackage{amsfonts} % needed for bold Greek, Fraktur, and blackboard bold
\usepackage{graphicx} % needed for figures
\usepackage{listings}

\begin{document}

% Be sure to use the \title, \author, \affiliation, and \abstract macros
% to format your title page.  Don't use lower-level macros to  manually
% adjust the fonts and centering.

\title{Fast Pulsed Short Solenoid Simulation
}
% In a long title you can use \\ to force a line break at a certain location.

\author{Alan DeSilva}
\email{hcarter333@tamu.edu} % optional
% optional second address
% If there were a second author at the same address, we would put another 
% \author{} statement here.  Don't combine multiple authors in a single
% \author statement.
\affiliation{Department of Physics, University of Maryland, College Park, MD 77843}
% Please provide a full mailing address here.

\author{Hamilton B. Carter}
\email{hcarter333@tamu.edu}
\affiliation{Department of Physics, Texas A\&M University, TX 77843}

% See the REVTeX documentation for more examples of author and affiliation lists.

\date{\today}

\begin{abstract}
This brief note  presents an updated version of DeSilva's can crusher simulator, whose results were first published 20 years ago in conjunction with the design of an associated laboratory demonstration.  The original simulator has been updated from IDL to Python running in the SageMath framework and open-sourced.  The code has also been re-structured into an object oriented architecture that allows for simple input of can crusher solenoid parameters.  This allows multiple solenoids with varying configurations to be simulated and compared.  We describe the facilities the updated simulator provides, along with the basics of its use.  While we're using the simulator to design apparatus for the study of bulk properties of superconducators, it can also be used conjunction with a can crusher apparatus as either part of an interactive demonstration laboratory.
\end{abstract}
% AJP requires an abstract for all regular article submissions.
% Abstracts are optional for submissions to the "Notes and Discussions" section.

\maketitle % title page is now complete


\section{Introduction} % Section titles are automatically converted to all-caps.
% Section numbering is automatic.

Our study of the dynamic behavior of rapidly quenched bulk superconductors has created a rather unique set of apparatus requirements.  In order to perform the experiment, a sample of superconducting Pb must be quenched by a magnetic field in less than 100 microseconds.  This requires a pulsed magnetic field of approximately 1 kG in magnitude with a transient pulse width of less than 100 microseconds.  For our purposes, the imploding soda can apparatus \textemdash also known as a can crusher \textemdash described by Alan DeSilva in his 1994 paper\cite{desilvacan} in this journal fit the bill perfectly.  The paper compared the results of a simulated apparatus with measurements made the actual apparatus  Dr. DeSilva was kind enough to locate the original IDL simulator code and forward it to us.  We have performed a language port, re-architected, and open sourced the simulator.  In Section II, we describe the language port and new object oriented structure of the simulator code.  In Section III, we describe how we have used the simulator for feasibility studies in our research.  We also provide code snippets and detailed instructions for using the simulator from any web browser.  Finally, in section IV, we conclude with a few thoughts on plans to use the simulator as part of an interactive laboratory demonstration.
\\
\\
\section{Language Port and Architecture}
While IDL is still a ubiquitous language, and a free version known as GDL can be installed on Linux platforms, we decided to port the code to Python executed within the SageMath platform.  Our team is familiar with Python and with the advent of packages like NumPy and SciPy, the language is gaining a certain noteriety with young physicists, SageMath has a cloud based implementation that has been funded by the National Science Foundation and Google. The user interface of this cloud-based software implementation is  accessible to the public through any web browser.  Consequently, anyone can make use of the SageMath platform and the ported simulator on any operting system that supports a web browser.  There are no operating system or software installation considerations to be made.
\\
\\
In addition to the language port, we re-architected the simulator to be object oriented.  While only the most rudimentary object architected structure has been put in place, it has already allowed us to simulate different solenoid parameters and their pertinence to our research with ease.  The basic object in the simulator is a member of the Crusher class.  Each Crusher object has its own set of parameters that fully describe a can crusher apparatus design configuration.  Among these configurable parameters  are the radius and the number of coils in the solenoid, the temperature sensitive resistivity of the solenoid, and the temperature of the solenoids operating environment.  As a concrete example, the following code sets up the Crusher object with a solenoid radius of $3.8 cm$, an initial power supply voltage of $5000\;\;Volts$, and an operating temperature of $4.2\;\;degrees\;\;K$.  the set\_movecan method tells the Crushcer object that we're simulating pulsing the magnetic field with a solid core in the solenoid's center, rather than a compressible soda can.
\\\\\\
\lstset{language=Python}
\begin{lstlisting}[frame=single]
load('./crushersim.sage')
#Set simulation time once
tickcount = 290
#This simulation calculates result for the coil size appropriate
#to the small YBCO sample
crushR = Crusher()
crushR.setr(3.8e-2)
crushR.setVnought(5000)
crushR.set_movecan(False)
crushR.setTemp(4.2)
crushR.simulate(tickcount)
#Find the maximum current

#Now plot the results ...
\end{lstlisting}
\section{Simulator Applications and Usage Instructions}
We created simulations that answered many questions about our experimental design without the expense of building and testing different pulsed magnetic solenoids.  Some concrete examples of these questions are: 
\\
1.  How does the transient response of the solenoid driving current change with respect to the operating temperature? This question was answered by configuring two Crusher objects with identical parameters excpet for the operating envrionment temperature.  This parpameter was changed using the objects setTemp method.  See figure 1 for results.
\\
2.  How does the maximum magnetic field available for a fixed initial power supply voltage change with respect to the radius of the solenoid? The initial setup was the same as in the first scenario; instantiate two Crusher objects with identical paramters.  The solenoid radii were then configured using the setr method.  Figure 2 shows the magnetic field with respet to distance from the center of the along the z axis --- transverse to the plane of the solenoid --- at a fixed radius incrementally smaller than the solenoid. 
\\
3.  The final example asks, "How does the pulse width  of the solenoid driving current change for solenoid designs with diferent radii? Figure 3 shows the results where once again, the radii of the two solenoids has been defined with the setr method. 
\\
\\
To make use of the simulator, simply perform the following steps.
\\
1.  Downlod the latest simulator release file crushersim.sage from \url{https://github.com/hcarter333/cancrusher} .  Also download one of the example simulations such as sim\_no\_sc\_42\_5000.sage
\\
2.  Create an account for yourself at \url{https://cloud.sagemath.com} .  Create a new project and uplaod both of the files from step 1 into the project.
\\
3.  Open the newly created project file sim\_no\_sc\_42\_5000.sage .  Click in the first cell, and then click on the 'Run' button in the upper left-hand corner of the interface.  After a short runtime, (on the order of thirty seconds depending on cloud load), you should see simulator results similar to the figures in this article.
\\
4.  Using the Wiki documentation avaiable at the github repostitory, \url{https://github.com/hcarter333/cancrusher/wiki}, customize the simulator as you please, and re-run.

\begin{figure}[page]
\centering
\includegraphics[width=5in]{42Current.PNG}
% Notice the width specification.  Photographs should normally have a
% resolution of approximately 300 pixels per inch when printed, that is,
% a total width of about 1000 pixels for a photo to be printed one column
% wide.  Note also that this included photo is in .jpg format even though 
% a .tiff version should be submitted for final production.
\caption{Current vs. time for the same solenoid at room temp and at liquid helium temperature.}
\label{sunsets}
\end{figure}
\begin{figure}[h!]
\centering
\includegraphics[width=5in]{HBigSmall42.PNG}
% Notice the width specification.  Photographs should normally have a
% resolution of approximately 300 pixels per inch when printed, that is,
% a total width of about 1000 pixels for a photo to be printed one column
% wide.  Note also that this included photo is in .jpg format even though 
% a .tiff version should be submitted for final production.
\caption{Field available for different solenoid radii corresponding to different supercondutor samples to be quenched.}
\label{sunsets}
\end{figure}
\begin{figure}[h!]
\centering
\includegraphics[width=5in]{bigsmallcurrent42.PNG}
% Notice the width specification.  Photographs should normally have a
% resolution of approximately 300 pixels per inch when printed, that is,
% a total width of about 1000 pixels for a photo to be printed one column
% wide.  Note also that this included photo is in .jpg format even though 
% a .tiff version should be submitted for final production.
\caption{Field available for different solenoid radii corresponding to different supercondutor samples to be quenched.}
\label{sunsets}
\end{figure}
\section{Conclusions}
The code form Dr. DeSilva's 1994 project has been ressurected and made freely available to anyone who would like to either use it or contribute to the project at \url{https://github.com/hcarter333/cancrusher}.  We have tested the feasiblity of several experimental design decisions using the simulator to produce a picture of how our pulsed magentic field apparatus will behave without having to actually construct equipment and perform time consuming characterizations.
\\
\\
We have plans to utilize the simulator in an online interactive demonstration laboratory\cite{ILD}.  In this small groups of students will be provided with workstations with the simulator pre-loaded into classroom SageMath accounts.  Based on the desribed parameters of the can crusher to be used in the demonstration, students will be asked to run a simulation and predict various measurable parameters of the demonstration such as the transient current measured using a pickup coil placed inside the can, or the temperature of the can immediately after a can crusher shot as measured by an infrared camera such as the Fluke Ti95.

\begin{acknowledgments}
The authors gratefully acknowledge helpful discussions with and suggestions from Alan DeSilva.  HBC was supported by a grant from the Texas Academy of Sciences.\end{acknowledgments}

\begin{thebibliography}{99}
% The numeral (here 99) in curly braces is nominally the number of entries in
% the bibliography. It's supposed to affect the amount of space around the
% numerical labels, so only the number of digits should matter--and even that
% seems to make no discernible difference.
% The issue number (3) in this citation is optional, because AJP's pagination 
% is by volume.

\bibitem{desilvacan} DeSilva, A. W., ``Magnetically imploded soft drink can", The American Journal of Physics, \textbf{62}, 41--45 (1994).  

\bibitem{ILD} Sokoloff, D. R. and Thornton, R. K., ``Using interactive lecture demonstrations to create an active learning environment", Phys. Teach. 35, 340--347 (1997).  



\end{thebibliography}

% If your manuscript is conditionally accepted, the editors will ask you to
% submit your editable LaTeX source file.  Before doing so, you should move
% all tables and figure captions to the end, as shown below.  Tables come 
% first, followed by figure captions (with figure inclusions commented-out).
% Figures should be submitted as separate files, collected with the
% LaTeX file into a single .zip archive.

%\newpage   % Start a new page for tables

%\begin{table}[h!]
%\centering
%\caption{Elementary bosons}
%\begin{ruledtabular}
%\begin{tabular}{l c c c c p{5cm}}
%Name & Symbol & Mass (GeV/$c^2$) & Spin & Discovered & Interacts with \\
%\hline
%Photon & $\gamma$ & \ \ 0 & 1 & 1905 & Electrically charged particles \\
%Gluons & $g$ & \ \ 0 & 1 & 1978 & Strongly interacting particles (quarks and gluons) \\
%Weak charged bosons & $W^\pm$ & \ 82 & 1 & 1983 & Quarks, leptons, $W^\pm$, $Z^0$, $\gamma$ \\
%Weak neutral boson & $Z^0$ & \ 91 & 1 & 1983 & Quarks, leptons, $W^\pm$, $Z^0$ \\
%Higgs boson & $H$ & 126 & 0 & 2012 & Massive particles (according to theory) \\
%\end{tabular}
%\end{ruledtabular}
%\label{bosons}
%\end{table}

%\newpage   % Start a new page for figure captions

%\section*{Figure captions}

%\begin{figure}[h!]
%\centering
%\includegraphics{GasBulbData.eps}   % This line stays commented-out
%\caption{Pressure as a function of temperature for a fixed volume of air.  
%The three data sets are for three different amounts of air in the container. 
%For an ideal gas, the pressure would go to zero at $-273^\circ$C.  (Notice
%that this is a vector graphic, so it can be viewed at any scale without
%seeing pixels.)}

%\label{gasbulbdata}
%\end{figure}

%\begin{figure}[h!]
%\centering
%\includegraphics[width=5in]{ThreeSunsets.jpg}   % This line stays commented-out
%\caption{Three overlaid sequences of photos of the setting sun, taken
%near the December solstice (left), September equinox (center), and
%June solstice (right), all from the same location at 41$^\circ$ north
%latitude. The time interval between images in each sequence is approximately
%four minutes.}
%\label{sunsets}
%\end{figure}

\end{document}
