\documentclass[prb,preprint]{revtex4-1} 
% The line above defines the type of LaTeX document.
% Note that AJP uses the same style as Phys. Rev. B (prb).

% The % character begins a comment, which continues to the end of the line.

\usepackage{amsmath}  % needed for \tfrac, \bmatrix, etc.
\usepackage{amsfonts} % needed for bold Greek, Fraktur, and blackboard bold
\usepackage{graphicx} % needed for figures

\begin{document}

% Be sure to use the \title, \author, \affiliation, and \abstract macros
% to format your title page.  Don't use lower-level macros to  manually
% adjust the fonts and centering.

\title{Fast Pulsed Short Solenoid Simulation
}
% In a long title you can use \\ to force a line break at a certain location.

\author{Alan DeSilva}
\email{hcarter333@tamu.edu} % optional
% optional second address
% If there were a second author at the same address, we would put another 
% \author{} statement here.  Don't combine multiple authors in a single
% \author statement.
\affiliation{Department of Physics, University of Maryland, College Park, MD 77843}
% Please provide a full mailing address here.

\author{Hamilton B. Carter}
\email{hcarter333@tamu.edu}
\affiliation{Department of Physics, Texas A\&M University, TX 77843}

% See the REVTeX documentation for more examples of author and affiliation lists.

\date{\today}

\begin{abstract}
We present an updated version of DeSilva's can crusher simulator, first used in 1994.  The original simulator has been updated from IDL to Python that runs in the SageMath framework.  The project has been open-sourced.  The code has alos been re-written in an object oriented architecture that allows for simple input of solenoid parameters.  This allows multiple solenoids with varying design envelopes to be simulated and compared.  We presesnt a brief note on the faciliteis the simulator now provids, and the basics of its use.  The simulator still serves perhaps most usefully in conjunction with a can crusher apparatus in an undergraduate laboratory setting.  In addition to the above, the code clearly points out the chain rule of differentiation and how it applies to dynamic experimetnal situations.
\end{abstract}
% AJP requires an abstract for all regular article submissions.
% Abstracts are optional for submissions to the "Notes and Discussions" section.

\maketitle % title page is now complete


\section{Instantaneous Rest Frames, Acceleration and Rapidity} % Section titles are automatically converted to all-caps.
% Section numbering is automatic.
Problems in special relativity are often met with trepidation by undergraduate and graduate physics students.  The solutions seem to have little resemblance to their classical counterparts when taken at face value, and students can be left with the feeling that they are relearning mechanics.  Problems that involve acceleration such as rocket propulsion and the twin paradox present even larger obstacles due to confusion about how to treat acceleration in the framework of special relativity.  While the simple one dimensional derivation of rocket velocity with respect to propellant mass appears in many undergraduate texts, its relativistic counterpart seldom does.  The twin paradox, resolved by taking into account that one of the twins is accelerating while the other is not, is ripe for misinterpretation as pointed out by Gruber \cite{gruberi}. These misinterpretations and general confusion about acceleration and the twin paradox led to the following amusing quote from Schild\cite{schildi}:

\begin{quote}
``A good many physicists believe that this paradox can only be resolved by the general theory of relativity.  They find great comfort in this, because they don't know any general relativity and feel that they don't have to worry about the problem until they decide to learn general relativity."
\end{quote}

We will show that the simple expression for rapidity can be utilized as a mathematical transform between the solution space of classical mechanics and that of special relativity.  This transform not only makes light work of the problems mentioned above, but it also places the student back in familiar territory by illustrating the relationship between familiar non-relativistic mechanics solutions and their special relativistic counterparts.   

The derivation of the transform requires a few definitions.  The rapidity of a particle moving with velocity   along the x –axis of the laboratory frame can be written as 
\begin{equation}
\label{rapidity}
\frac{v}{c} = tanh\left(\frac{w}{c}\right).
\end{equation}
where $c$ is the speed of light.  In most texts, the rapidity is denoted simply as $w$ rather than $w/c$.  We have chosen to scale the rapidity by $1/c$ to simplify our further derivations.  Karapetoff, who wrote one of the most extensive treatments of rapidity\cite{karapetoffi} also noted that 
\begin{equation}
\label{gamma}
\gamma=\frac{dt}{d\tau} = cosh\left(\frac{w}{c}\right).
\end{equation}
where $t$ denotes laboratory frame time, and $\tau$ denotes proper time.  

The final expression we require relates an infinitesimal change in velocity, $dv$, as measured in the laboratory frame to an infinitesimal increment in velocity, $dv'$, as measured between two successive instantaneous reference frames, (IRFs).  Rindler gives this expression\cite{rindlerii} as
\begin{equation}
\label{gsquared}
dv=\gamma^{-2}\left(v\right)dv'.
\end{equation}
The expression can be derived by first expressing $dv$ as acceleration times an incremental step in time, then Lorentz transforming the product to the moving frame to give
\begin{eqnarray}
\label{gsquaredderiv}
dv& = &\left(\frac{d^2x}{dt^2}\right)dt = \left(\frac{\left(1/\gamma\right)d^2x'}{\left(\gamma^2\right)d\tau^2}\right)\left(\gamma\right)d\tau = \gamma^{-2}dv'.
\end{eqnarray}

It will now be shown that $w$, the rapidity, defined above in equation (\ref{rapidity}), corresponds to the velocity defined by linearly accumulating the infinitesimal changes in the IRF velocity, $dv'$.  This accumulated non-relativistic velocity  is the same velocity that would be measured by an observer moving with the particle using an accelerometer in conjunction with a stopwatch.   By referencing the accelerometer at each tick of the clock, the co-moving observer can calculate $v'$ as a function of proper time:
\begin{equation}
\label{accint}
v'\left(\tau\right)=\int_0^{\tau} \! a\left(\tau'\right)d\tau'+v'\left(0\right),
\end{equation}
where $a\left(\tau\right)$ is the proper acceleration, and $v'\left(0\right)$ is the object's initial velocity.  We have chosen v'(0) to be zero with no loss of generality as we shall see.  Using equations (\ref{rapidity}), (\ref{gsquared}), and (\ref{accint}), it can be shown that 
\begin{equation}
\frac{dv}{dv'}=\gamma^{-2}\left(v\right)=\frac{1}{cosh^2\left(\frac{w}{c}\right)}=\frac{dv}{dw},
\end{equation}
which implies
\begin{equation}
\label{rapnonrel}
dw=dv'
\end{equation}

In other words, the oft-mentioned, but rarely expanded-upon rapidity is simply a scaled version of the, (generally time-dependent), non-relativistic velocity $v'$ with respect to the laboratory frame!

Equation (\ref{rapnonrel}) can be integrated to give
\begin{equation}
\label{rapinit}
w=v'+w\left(0\right)
\end{equation}
where $w\left(0\right)$ is an initial rapidity that is related to the moving particle's initial laboratory frame velocity by 
\begin{equation}
\label{raptanh}
\frac{w\left(0\right)}{c}=tanh^{-1}\left(\frac{v\left(0\right)}{c}\right)
\end{equation}

\section{Examples}

With our new-found interpretation of rapidity in hand, let's try a few examples.  Our first example provides a simple derivation of the hyperbolic equations of motion for a uniformly accelerated particle and also illuminates another important aspect of rapidity that follows from the derivation shown above.  In Rindler's original derivation of hyperbolic motion\cite{rinlderi} he calculated the distance traveled by a rocket in the laboratory frame with respect to the travel time as perceived by the passengers of the rocket, this is known as the proper velocity and is denoted by  $dx/d\tau$.  We will re-derive Rindler's expression for proper velocity using the rapidity transform shown above.  The non-relativistic expression for the velocity of a particle starting from rest at the origin and moving with constant acceleration $\alpha$ with respect to the rest frame of the particle is
\begin{equation}
\label{rindtransformI}
v'=\alpha\tau=w.
\end{equation}
Inserting this into our transform gives
\begin{equation}
\label{transform1}
\frac{v}{c} = tanh\left(\frac{v'}{c}\right)=tanh\left(\frac{\alpha \tau}{c}\right).
\end{equation}
This can be written as
\begin{equation}
\label{rapidityexp}
\frac{1}{c}\frac{dx}{dt} = tanh\left(\frac{\alpha \tau}{c}\right)=\frac{sinh\left(\frac{\alpha \tau}{c}\right)}{cosh\left(\frac{\alpha \tau}{c}\right)}.
\end{equation}
Multiplying both sides by the hyperbolic cosine in the denominator and using the substitution defined by (\ref{gamma}) we get 
\begin{equation}
\frac{1}{c}\frac{dx}{dt}cosh\left(\frac{\alpha \tau}{c}\right) = \frac{1}{c}\frac{dx}{dt}\frac{dt}{d\tau}=\frac{1}{c}\frac{dx}{d\tau}= sinh\left(\frac{\alpha \tau}{c}\right),
\end{equation}
the desired expression for $dx/d\tau$.  To get the distance traveled with respect to $\tau$ we integrate to get
\begin{equation}
x=\frac{c^2}{\alpha}cosh\left(\frac{\alpha\tau}{c}\right)-\frac{c^2}{\alpha},
\end{equation}
which is identical to Rindler's result\cite{rinlderi}.  A similar derivation can be performed to arrive at
\begin{equation}
t =\frac{c}{\alpha}sinh\left(\frac{\alpha\tau}{c}\right)
\end{equation}
which can be used to explore the twin paradox.

The following pair of equations can be obtained easily from (\ref{rapidityexp}) 
\begin{eqnarray}
\frac{dx}{d\tau} & = & c\ sinh\left(\frac{v'\left(\tau\right)}{c}\right) \nonumber \\
\frac{d\left(ct\right)}{d\tau} & = & c\ cosh\left(\frac{v'\left(\tau\right)}{c}\right).
\end{eqnarray}
These two equations demonstrate that classical velocities can be projected (see Figure~\ref{sunsets}) directly onto the two-dimensional Minkowski space-time axes representing the object's ‘proper speeds’ through laboratory space and time respectively.  The object's proper speed with respect to space is the distance traveled in the laboratory frame with respect to the object's proper time, as defined by Brehme\cite{brehmefour}.  The object's proper speed through laboratory time is $dt/d\tau=\gamma$.

\begin{figure}[h!]
\centering
\includegraphics[width=5in]{hypproj4.png}
% Notice the width specification.  Photographs should normally have a
% resolution of approximately 300 pixels per inch when printed, that is,
% a total width of about 1000 pixels for a photo to be printed one column
% wide.  Note also that this included photo is in .jpg format even though 
% a .tiff version should be submitted for final production.
\caption{Projection of Classical Velocity into Minkowski Velocity Space}
\label{sunsets}
\end{figure}

For a complete treatment of proper velocity, we refer the reader to Robert Flynn's excellent primer on spacecraft navigation\cite{flynn}.  Note that it can immediately be seen that as the object's proper speed increases, the rate of progression of laboratory time with respect to proper time also increases due to time dilation.  In other words, as the object's relative speed through laboratory space increases, its relative speed through laboratory time decreases as pointed out by Brehme\cite{brehmegeo} and popularized by Greene\cite{greene}.

In the introduction, reference was made to the relativistic rocket propulsion problem.  Goldstein, in his graduate mechanics text\cite{goldsteini}, begins his version of the problem by noting that ``The theory of rocket motion developed in Exercise 3, Chapter 1, no longer applies in the relativistic region, in part because there is no longer conservation of mass."  He then asks the student to show that a different differential equation holds in the relativistic regime,
\begin{equation}
m\frac{dv}{dm}+a\left(1-\frac{v^2}{c^2}\right)=0,
\end{equation}
and to show that the solution to this new equation can be written as 
\begin{equation}
\frac{v}{c}=\frac{\left(1-\frac{m}{m_o}\right)^{\frac{2a}{c}}}{\left(1+\frac{m}{m_o}\right)^{\frac{2a}{c}}}
\end{equation}

Using the transformation method illustrated here, no new differential equation is necessary.  We simply transform the solution of the non-relativistic rocket problem for a rocket starting from rest, 
\begin{equation}
v=a\ ln\left(\frac{m_o}{m_f}\right)
\end{equation}
where $a$ is the exhaust velocity of the propellant with respect to the rocket,   $m_o$ is the initial mass of the rocket and $m_f$ is the final mass of the rocket.  The transform yields 
\begin{equation}
\frac{v}{c}=tanh\left(\frac{a}{c} ln\left(\frac{m_o}{m_f}\right)\right).
\end{equation}
Keeping in mind that the hyperbolic tangent can be written as
\begin{equation}
tanh\left(u\right)=\frac{e^u-e^{-u}}{e^u+e^{-u}},
\end{equation}
a short bit of algebra shows
\begin{equation}
\frac{v}{c}=\frac{\left(1-\frac{m_f}{m_o}\right)^{\frac{2a}{c}}}{\left(1+\frac{m_f}{m_o}\right)^{\frac{2a}{c}}},
\end{equation}
as indicated by Goldstein.

\section{Conclusion}

We've shown that the unassuming expression for rapidity can be used to transform solutions of kinematical problems from non-relativistic mechanics directly to their relativistic counterparts.  In addition to making easy work of relativistic mechanics, the method serves as a useful pedagogical tool, demonstrating that relativistic solutions can be seen as a direct consequence of their classical progenitors.  By separating the rapidity expression into its hyperbolic sine and cosine terms, it was shown that classical velocities can be projected directly onto their proper velocity counterparts.  This projection serves to further establish an intuitive relationship between classical and relativistic mechanics and also provides an effective springboard for discussions of the hyperbolic geometric nature of Minkowski space-time.

\begin{acknowledgments}
The authors gratefully acknowledge helpful discussions with and suggestions from George Goedecke and Stephen Fulling.  HBC was supported by NSF Grant \#0968269.\end{acknowledgments}

\begin{thebibliography}{99}
% The numeral (here 99) in curly braces is nominally the number of entries in
% the bibliography. It's supposed to affect the amount of space around the
% numerical labels, so only the number of digits should matter--and even that
% seems to make no discernible difference.
% The issue number (3) in this citation is optional, because AJP's pagination 
% is by volume.

\bibitem{desilvacan} DeSilva, A. W., ``Magnetically imploded soft drink can", The American Journal of Physics, \textbf{62}, 41--45 (1994).  



\end{thebibliography}

% If your manuscript is conditionally accepted, the editors will ask you to
% submit your editable LaTeX source file.  Before doing so, you should move
% all tables and figure captions to the end, as shown below.  Tables come 
% first, followed by figure captions (with figure inclusions commented-out).
% Figures should be submitted as separate files, collected with the
% LaTeX file into a single .zip archive.

%\newpage   % Start a new page for tables

%\begin{table}[h!]
%\centering
%\caption{Elementary bosons}
%\begin{ruledtabular}
%\begin{tabular}{l c c c c p{5cm}}
%Name & Symbol & Mass (GeV/$c^2$) & Spin & Discovered & Interacts with \\
%\hline
%Photon & $\gamma$ & \ \ 0 & 1 & 1905 & Electrically charged particles \\
%Gluons & $g$ & \ \ 0 & 1 & 1978 & Strongly interacting particles (quarks and gluons) \\
%Weak charged bosons & $W^\pm$ & \ 82 & 1 & 1983 & Quarks, leptons, $W^\pm$, $Z^0$, $\gamma$ \\
%Weak neutral boson & $Z^0$ & \ 91 & 1 & 1983 & Quarks, leptons, $W^\pm$, $Z^0$ \\
%Higgs boson & $H$ & 126 & 0 & 2012 & Massive particles (according to theory) \\
%\end{tabular}
%\end{ruledtabular}
%\label{bosons}
%\end{table}

%\newpage   % Start a new page for figure captions

%\section*{Figure captions}

%\begin{figure}[h!]
%\centering
%\includegraphics{GasBulbData.eps}   % This line stays commented-out
%\caption{Pressure as a function of temperature for a fixed volume of air.  
%The three data sets are for three different amounts of air in the container. 
%For an ideal gas, the pressure would go to zero at $-273^\circ$C.  (Notice
%that this is a vector graphic, so it can be viewed at any scale without
%seeing pixels.)}

%\label{gasbulbdata}
%\end{figure}

%\begin{figure}[h!]
%\centering
%\includegraphics[width=5in]{ThreeSunsets.jpg}   % This line stays commented-out
%\caption{Three overlaid sequences of photos of the setting sun, taken
%near the December solstice (left), September equinox (center), and
%June solstice (right), all from the same location at 41$^\circ$ north
%latitude. The time interval between images in each sequence is approximately
%four minutes.}
%\label{sunsets}
%\end{figure}

\end{document}
